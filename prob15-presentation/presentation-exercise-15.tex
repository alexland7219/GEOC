\documentclass{beamer}
\usepackage{ragged2e}
\usepackage{graphicx}
\usepackage{xcolor}
\usepackage{algorithm}
\usepackage[noend]{algpseudocode}
% Theme
\usetheme[block=fill, progressbar=frametitle]{metropolis} % Choose a theme (e.g., "Madrid", "Warsaw", "CambridgeUS", "Boadilla", etc.)
\usecolortheme{spruce}

\setbeamercolor{background canvas}{bg=white}
% Title
\title{Rectangular Partitioning of Rectilinear Polygons}
\subtitle{Exercise 15}
\author{Alexandre Ros}
\date{\today}

% Colors
\definecolor{concave}{RGB}{160,32,240} % Define a custom color named "mycolor"
\definecolor{convex}{RGB}{139,0,0} % Define a custom color named "mycolor"
\definecolor{filling}{RGB}{72,116,94} % Define a custom color named "mycolor"

% Begin the document
\begin{document}

% Title page
\frame{\titlepage}

% Table of Contents
% \begin{frame}
%    \frametitle{Table of Contents}
%    \tableofcontents
%\end{frame}

% Section 1
\section{Introduction to the problem}

\begin{frame}{Introduction}
	\begin{block}{Exercise 15}
	\textbf{Decompose a rectilinear polygon into rectangles}, using segments aligned with the edges of the polygon.
The algorithm must produce the \textbf{minimum number of pieces}, and the output must give a
complete description of the partition.
	\end{block}
	\begin{block}{Input}
	A polygon whose sides meet at right angles.
	\end{block}  
	\begin{block}{Output}
	A \textbf{partition} of the polygon into rectangles, with no overlaps.
	\end{block}
	
	We will call this problem $MNC$ (minimal nonoverlapping cover).
\end{frame}

\begin{frame}[t, shrink=10]{Examples}
\vspace{5px}

\centering
  \includegraphics[width=\textwidth]{"./rect1.png"}
  
  \vspace{15px}
	Multiple partitions may exist
\end{frame}


% Section 2
\section{Preliminaries}

\begin{frame}[t]{Concave and convex vertices}
	In a rectilinear polygon, we can distinguish between concave and convex vertices.
	
    \begin{itemize}
        \item A vertex is said to be \textcolor{concave}{\textbf{concave}} if its interior angle is over 180°.
        \item A vertex is said to be \textcolor{convex}{\textbf{convex}} if its interior angle is under 180°.
    \end{itemize}
    \vspace{10px}
\begin{figure}
\centering
  \includegraphics[width=.6\textwidth]{"./concave.png"}
  \caption{Convex (red) and concave (purple) vertices.}
     \label{fig:question}
\end{figure}

\end{frame}

\begin{frame}[t]{Co-vertical, co-horizontal and chords}	
    \begin{itemize}
        \item Two vertices $(x_1, y_1)$, $(x_2, y_2)$ that share no edge are \textbf{co-vertical} $\iff y_1 = y_2$ and \textbf{co-horizontal} $\iff x_1 = x_2$. 
        \item A \textbf{chord} is a segment fully contained inside the polygon that connects two co-horizontal or co-vertical vertices.
    \end{itemize}
    \vspace{10px}
\begin{figure}
\centering
  \includegraphics[width=.6\textwidth]{"./chords.png"}
  \caption{Examples of chords (orange) and \textbf{not} chords (red).}
     \label{fig:question}
\end{figure}

\end{frame}


\section{Chord-less polygons}

\begin{frame}[t]{Concave vertices observations}	
	Suppose, for now, that there are no chords in polygon $P$.

	\begin{block}{Observation \#1}
	$P$ is a rectangle $\iff$ $P$ has exactly four convex vertices.
	\end{block}  

	\begin{block}{Observation \#2}
	A concave vertex must be a vertex of two rectangles of the partition. See example below.
	\end{block}  
\begin{figure}

\centering
  \includegraphics[width=.6\textwidth]{"./direccio.png"}
  \caption{Concave vertex.}
     \label{fig:question}
\end{figure}


\end{frame}

\begin{frame}[t]{Algorithm for chord-less polygons}	
	\begin{block}{Observation \#3}
	If $P$ is not a rectangle then it must have at least one concave vertex.
	\end{block}  

	\begin{block}{Idea to solve the problem}
	For each concave vertex, extend its vertical edge until it hits another edge.
	\textit{If it hits a vertex, then that extension is a chord!}
	\end{block}  
\begin{figure}

\centering
  \includegraphics[width=.5\textwidth]{"./launchrays.png"}
  \caption{Chord-less polygons optimal solution.}
     \label{fig:question}
\end{figure}

\end{frame}

\begin{frame}[t]{Algorithm for chord-less polygons}	

  \begin{algorithm}[H] % H option forces the algorithm to stay in place
    \caption{Chord-less MNC in $O(n \log n)$}
    \begin{algorithmic}
      % Your algorithm code goes here
      \Procedure{MNC}{$P$}
      \State $E \leftarrow \emptyset$; $R \leftarrow \emptyset$
        \State $L \leftarrow $ sorted vertices of $P$ by x-coordinate¹.
        \For{each vertex $v$ in $P$}
        \State Insert $v$ to $E$ in position, delete position if duplicate.
        \If{$v$ is concave}
        	  \State Extend vertical edge.
          \State Insert to $R$ the vertical edge.
          \EndIf
        \EndFor
        \State Return $R$ as the paritioning edges of $P$.
      \EndProcedure
    \end{algorithmic}
  \end{algorithm}	
	¹ If using y-coordinate as tiebreaker, we can read pairs of vertices at a time (vertical edges). 

\end{frame}

\foreach \i in {0,1,2,3,4,5,6,7,8}{
  \begin{frame}[t]{Algorithm for chord-less polygons}
    \centering
    \includegraphics[width=\textwidth]{sweep\i.png}
  \end{frame}
}

\section{Polygons with chords}


\begin{frame}[t]{Polygons with chords}	
	\begin{block}{Problem \#1}
	Suppose there are two co-horizontal vertices which form a chord.
	We may prefer joining that chord rather than extending the vertical edge.
	\end{block}  
	
    \centering
    \includegraphics[width=0.7\textwidth]{problem.png}

	We would end up with one more rectangle!

\end{frame}

\begin{frame}[t]{Polygons with chords}	
	\begin{block}{Problem \#2}
	If we drew all chords, there may be intersections.
	\end{block}  
	\vspace{10px}
    \centering
    \includegraphics[width=0.6\textwidth]{intersect0.png}

	Chords $ab$ and $ch$ intersect. So do chords $ij$ and $di$.

\end{frame}

\begin{frame}[t]{Polygons with chords}	
	\begin{block}{Solution}
	Draw the largest set of non-intersecting chords. 
	After this step, no chords can remain.
    The remaining sub-polygons can be partitioned as usual.  

	\end{block}  
	\vspace{5px}
    \centering
    \includegraphics[width=0.6\textwidth]{intersect1.png}

\end{frame}


\begin{frame}[t]{Polygons with chords}	
	\begin{block}{Theorem (Ferrari, 1984)}
	A rectilinear polygon $R$ has a minimum partition of order $N - B + 1$, where
	
	
	$N = $ Total number of concave vertices on the boundary of $R$.
	$B = $ Maximum number of nonintersecting chords.

	\end{block}  
	\centering
    \includegraphics[width=0.43\textwidth]{intersect1.png}
    
    $N = 10$ and $B = 4$ therefore the optimal order is 7.

\end{frame}

\begin{frame}[t]{Finding the largest set of nonintersecting chords}	
	\begin{block}{Process}
	Construct a bipartite graph $B$, with vertices $v_i$ for vertical chords, $h_i$ for horizontal chords, and edges $\{v_i, h_j\}$ if $v_i$ and $h_j$ intersect.
	
	Problem reduces to finding the maximum independent set of $B$.

	
	\end{block}
	\centering
    \includegraphics[width=0.9\textwidth]{graph0.png}

\end{frame}

\begin{frame}[t]{Finding the largest set of nonintersecting chords}	
	\begin{block}{Maximum Independent Set}
	Recall that the MIS problem for a general graph is NP-hard. 	
	It is possible to find the MIS of a \textbf{bipartite} graph in polynomial time using Kőnig's theorem.
	
	\end{block}

	\begin{block}{Kőnig's theorem}
    In any bipartite graph, the number of edges in a maximum matching equals the number of vertices in a minimum vertex cover.
    	
	\end{block}

	\begin{block}{Min Vertex Cover / Max Independent Set}
	
	Both problems are complements of each other. So, the number of edges in a max matching is $\#V$ minus the size of the MIS.	
	    	
	\end{block}

\end{frame}

\begin{frame}[t, shrink=10]{Max. Independent Set on Bipartite Graphs}

Hopcroft-Karp algorithm for finding a maximum matching on a bipartite graph, in $O(n^{2.5})$, being $n$ the number of vertices.

	Every edge in the matching will have one vertex in the VC and one in IS. 

	Knowing the size of the MIS, computing the actual set can be accomplished in $\omega(n^{2.5})$ using a simple algorithm.
	\vspace{10px}

	\centering
    \includegraphics[width=0.8\textwidth]{mis.png}


\end{frame}

\begin{frame}[t, shrink=10]{Recap}

	\begin{block}{Algorithm - $O(n^{2.5})$}
	1. Find chords of $R$.
	
	2. Construct the bipartite graph $B = (V, H, E)$ as follows: each vertex $v_i$ in $V$ corresponds to a vertical chord, every $h_i$ in $H$ to a horizontal chord, and each edge $v_i h_j$ in $E$ corresponds to an intersection between $v_i$ and $h_i$.

	3. Find maximum matching $M$ of $B$. $O(n^{2.5})$
	
	4. Find maximum independent set $S$ of $B$ based on $M$, by using Kőnig's Theorem. Let $b = |S|$.

	5. Draw $b$ chords corresponding to $S$, dividing $R$ into $b + 1$ subpolygons, with each subpolygon being chord-less.
	
	6. A minimal partition of each subpolygon can be found by using the sweep-line algorithm provided in slide 8.
		\end{block}
\end{frame}


\begin{frame}[t]{Holes}

	The problem can be solved if it has holes. In that case, the smallest number of rectangles in a rectangular partition of a nonsimply rectilinear polygon $R$ is:

	\begin{center}
		$N - B + 1 - D$, being 
	\end{center}
	
	$N$ = Total number of concave vertices contained inside $R$. 

	$L$ = Maximum number of nonintersecting chords in $R$. 

	$D$ = Number of holes in $R$. 

\end{frame}

\begin{frame}{Improvements and References}

For a polygon with holes, the optimal is $\Omega(n \log n)$.

It is possible to solve the problem without constructing the graph in
$O(n)$ (Liou et al, 1990).

The 3D version using orthogonal parallelepipeds is NP-complete.


\textbf{References}

  \begin{itemize}
    \item Liou et al, \emph{Minimum Rectangular Partition Problem for Simple
Rectilinear Polygons}, IEEE Trans. Computer-Aided Design, 1990.
    \item Ferrari et al, \emph{Minimal Rectangular Partitions of Digitized Blobs}, Department of Electrical Engineering University of California, 1984.
    % Add more references as needed
  \end{itemize}
\end{frame}


% Thank you slide
\begin{frame}
    \frametitle{Thank You}
    \centering
    \Huge Thanks for your attention!
\end{frame}

\end{document}
